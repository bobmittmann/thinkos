{\includegraphics{images/image1.png}}{~ }

{ThinkOS demo on Windows}

\begin{itemize}
\item
  \hypertarget{h.gjdgxs}{}

  {For embedded developers on a budget.}
\item
  {All open source and free of}{~}{charge}{~}{tools.}
\end{itemize}

\section{\texorpdfstring{{Quick Guide}}{Quick Guide}}\label{quick-guide}

{This guide will help you to install the necessary tools to compile and
run a }{ThinkOS}{~demo application on a development board.}

\subsection{\texorpdfstring{{Windows
Tools}}{Windows Tools}}\label{windows-tools}

{These are the basic steps to follow in order to install the tools: }

\begin{enumerate}
\tightlist
\item
  {Install }{\protect\hyperlink{h.30j0zll}{GNU Toolchain for ARM
  Processros}}{\protect\hyperlink{h.30j0zll}{~}}{for Windows hosts.}
\item
  {Install }{\protect\hyperlink{h.1fob9te}{GNU
  Make}}{\protect\hyperlink{h.1fob9te}{~}}{for Windows.}
\item
  {Install }{\protect\hyperlink{h.3znysh7}{GNU
  CoreUtils}}{\protect\hyperlink{h.3znysh7}{~}}{for Windows.}
\item
  {Install the }{\protect\hyperlink{h.2et92p0}{Java
  JRE}}{\protect\hyperlink{h.2et92p0}{~}}{from Oracle.}
\item
  {Install the }{\protect\hyperlink{h.tyjcwt}{Eclipse for C++
  Developers}}{\protect\hyperlink{h.tyjcwt}{~}}{IDE.}
\item
  {Install the }{\protect\hyperlink{h.3dy6vkm}{GNU ARM Eclipse
  Plug-ins}}{\protect\hyperlink{h.3dy6vkm}{.}}
\item
  {Install the }{\protect\hyperlink{h.1t3h5sf}{C/C++ GDB Hardware
  Debugging Plug-in}}{\protect\hyperlink{h.1t3h5sf}{.}}
\item
  {Install the }{\protect\hyperlink{h.4d34og8}{STM32 ST-LINK
  Utility}}{\protect\hyperlink{h.4d34og8}{~}}{from STMicroelectronics.}
\end{enumerate}

{Detailed instructions for each one of the steps can be found in the
next sections of this document. }

\subsection{\texorpdfstring{{Demo
Application}}{Demo Application}}\label{demo-application}

{There are some demo applications using ThinkOS in the source tree. This
is a guide on how to compile and run }{ThinkOS}{~programs in the
}{\protect\hyperlink{h.35nkun2}{STM32F3Discovery}}{\protect\hyperlink{h.35nkun2}{~}}{board.
}

\begin{enumerate}
\tightlist
\item
  {Get the }{\protect\hyperlink{h.2s8eyo1}{ThinkOS Source
  Code}}{\protect\hyperlink{h.2s8eyo1}{~}}{from GitHub.}
\item
  {\protect\hyperlink{h.17dp8vu}{Import the Eclipse
  projects}}{\protect\hyperlink{h.17dp8vu}{~}}{for the target board into
  your workspace.}
\item
  {\protect\hyperlink{h.3rdcrjn}{Compile the ThikOS
  BootLoader}}{\protect\hyperlink{h.3rdcrjn}{~}}{project using Eclipse
  IDE.}
\item
  {Load the }{\protect\hyperlink{h.1ksv4uv}{ThinkOS
  BootLoader}}{\protect\hyperlink{h.1ksv4uv}{~}}{on the board using the
  }{STM32 ST-LINK Utility}{.}
\item
  {Install the }{\protect\hyperlink{h.44sinio}{Virtual COM
  Driver}}{\protect\hyperlink{h.44sinio}{~}}{on the PC to access the
  }{ThinkOS Debug-Monitor}{.}
\item
  {\protect\hyperlink{h.26in1rg}{Compile the SDK
  Libraries}}{\protect\hyperlink{h.26in1rg}{~}}{(sdklib) for the target
  board using Eclipse.}
\item
  {\protect\hyperlink{h.lnxbz9}{Compile the demo
  application}}{\protect\hyperlink{h.lnxbz9}{~}}{project using Eclipse.}
\item
  {Run the application with }{\protect\hyperlink{h.2jxsxqh}{GDB Remote
  Debugging}}{\protect\hyperlink{h.2jxsxqh}{.}}
\end{enumerate}

\hypertarget{h.30j0zll}{\section{\texorpdfstring{{GNU Toolchain for ARM
Processors}}{GNU Toolchain for ARM Processors}}\label{h.30j0zll}}

{The GNU toolchain is a broad collection of programming tools produced
by the GNU Project. These tools form a toolchain (a suite of tools used
in a serial manner) used for developing software applications and
operating systems.}

\begin{itemize}
\tightlist
\item
  {Download and install the }{GNU Toolchain for ARM Processros}{~from
  the site:
  }{\href{https://www.google.com/url?q=https://launchpad.net/gcc-arm-embedded\&sa=D\&ust=1511275046369000\&usg=AFQjCNGxWGT_YonUxM4BFF9geho0LfeKww}{https://launchpad.net/gcc-arm-embedded}}{.
  At the time of writing this document the latest version was
  4.9-2015-q2, which can be downloaded directly from:
  }{\href{https://www.google.com/url?q=https://launchpad.net/gcc-arm-embedded/4.9/4.9-2015-q2-update/\%2Bdownload/gcc-arm-none-eabi-4_9-2015q2-20150609-win32.exe\&sa=D\&ust=1511275046369000\&usg=AFQjCNHgw7yb_djkKsPe-rUggiSx6ZpV-Q}{https://launchpad.net/gcc-arm-embedded/4.9/4.9-2015-q2-update/+download/gcc-arm-none-eabi-4\_9-2015q2-20150609-win32.exe}}
\item
  {Run the downloaded file
  (gcc-arm-none-eabi-4\_9-2015q2-20150609-win32.exe) to start the
  installation.}
\end{itemize}

{\includegraphics{images/image3.png}}

\begin{itemize}
\tightlist
\item
  {In the next steps use the default values suggested by the wizard
  installer.}
\item
  {In the last screen, after the files were copied, select the ``Add
  path to environment variable'' option. }
\end{itemize}

{\includegraphics{images/image5.png}}

\hypertarget{h.1fob9te}{\section{\texorpdfstring{{GNU Make for
Windows}}{GNU Make for Windows}}\label{h.1fob9te}}

{Make is a tool which controls the generation of executables and other
non-source files of a program from the program's source files. Make gets
its knowledge of how to build your program from a file called the
makefile, which lists each of the non-source files and how to compute it
from other files. When you write a program, you should write a makefile
for it, so that it is possible to use Make to build and install the
program.}

\begin{itemize}
\tightlist
\item
  {Download the GNU Make package:
  http://sourceforge.net/projects/gnuwin32/files/make/3.81/make-3.81.exe/download}
\item
  {Run the downloaded program ``make-3.81.exe'' and follow the
  installation steps. }
\end{itemize}

{\includegraphics{images/image4.png}}

\begin{itemize}
\tightlist
\item
  {Install on the same folder as the GCC Toolchain ~so you won't need to
  modify your path. During the installation in the ``Select Destination
  Location:'' page select: ``C:\textbackslash{}Program Files
  (x86)\textbackslash{}GNU Tools ARM Embedded\textbackslash{}4.9
  2015q2'' in the text box.}
\end{itemize}

{\includegraphics{images/image7.png}}

\begin{itemize}
\tightlist
\item
  {Also there is no point of creating a ``Start Menu'' folder. Disable
  this option by checking the ``Don't create a Start Menu folder'' in
  the page: ``Select Start Menu Folder''. }
\end{itemize}

\hypertarget{h.3znysh7}{\section{\texorpdfstring{{GNU CoreUtils for
Windows}}{GNU CoreUtils for Windows}}\label{h.3znysh7}}

{The GNU Core Utilities are the basic file, shell and text manipulation
utilities of the GNU operating system. These are the core utilities
which are expected to exist on every operating system.}

\begin{itemize}
\tightlist
\item
  {Download the GnuCore Utils package from:
  }{\href{https://www.google.com/url?q=http://sourceforge.net/projects/gnuwin32/files/coreutils/5.3.0/coreutils-5.3.0.exe/download\&sa=D\&ust=1511275046372000\&usg=AFQjCNEiYiOPmJnrJ9jTIPo2YGpR3Yz9wg}{http://sourceforge.net/projects/gnuwin32/files/coreutils/5.3.0/coreutils-5.3.0.exe/download}}
\item
  {Run the downloaded program ``coreutils-5.3.0.exe'' and follow the
  installation steps.}
\item
  {Please refer to the }{GNU Make for Windows}{~installation procedure.}
\end{itemize}

\hypertarget{h.2et92p0}{\section{\texorpdfstring{{Java Runtime
Environment (JRE)}}{Java Runtime Environment (JRE)}}\label{h.2et92p0}}

{Eclipse depends on a Java so a Java Runtime Environment must be
installed in your system.}

\subsection{\texorpdfstring{{Do I have Java in my
system?}}{Do I have Java in my system?}}\label{do-i-have-java-in-my-system}

{The easy way to check whether you have the JRE installed in your system
is to type call the java interpreter from a windows console.}

\begin{itemize}
\tightlist
\item
  {Run the command shell: ``Start''--\textgreater{}``All
  Programs''--\textgreater{}``Accessories''--\textgreater{}``Command
  Prompt''}
\item
  {Type: ``java -version''}
\item
  {The output should be something similar to this:}
\end{itemize}

{java version ``1.8.0\_40''}

{Java(TM) SE Runtime Environment (build 1.8.0\_40-b26)}

{Java HotSpot(TM) 64-Bit Server VM (build 25.40-b25, mixed mode)}

\subsection{\texorpdfstring{{Installing the
JRE}}{Installing the JRE}}\label{installing-the-jre}

\begin{itemize}
\tightlist
\item
  {Download the JRE from:
  }{\href{https://www.google.com/url?q=http://www.oracle.com/technetwork/java/javase/downloads/index.html\&sa=D\&ust=1511275046373000\&usg=AFQjCNFFJIGcAOFgmFHQHLk6g_3Ml19LyA}{http://www.oracle.com/technetwork/java/javase/downloads/index.html}}{.}
\item
  {Select the correct version for your platform. Most probably the Java
  version you want is the 64 bits version for ~Windows hosts: "}{Windows
  x64}{" --\textgreater{} "}{jre-8u60-windows-x64.exe}{"}
\end{itemize}

\hypertarget{h.tyjcwt}{\section{\texorpdfstring{{Eclipse IDE for C/C++
Developers}}{Eclipse IDE for C/C++ Developers}}\label{h.tyjcwt}}

\begin{itemize}
\tightlist
\item
  {Download the }{Eclipse IDE for C/C++ Developers}{~from
  }{\href{https://www.google.com/url?q=https://eclipse.org/downloads/\&sa=D\&ust=1511275046374000\&usg=AFQjCNGku66E543IPyqaPfqFKQo3pd2L3A}{https://eclipse.org/downloads/}}{.
  At the time of this writing the latest version is }{Mars}{~wich can be
  downloaded directly from:
  }{\href{https://www.google.com/url?q=http://www.eclipse.org/downloads/download.php?file\%3D/technology/epp/downloads/release/mars/R/eclipse-cpp-mars-R-win32-x86_64.zip\&sa=D\&ust=1511275046375000\&usg=AFQjCNGpm_xGDGG0CV-OUsVfXgLuVmq5zw}{http://www.eclipse.org/downloads/download.php?file=/technology/epp/downloads/release/mars/R/eclipse-cpp-mars-R-win32-x86\_64.zip}}
\item
  {Unzip the downloaded file into a directory of your choice. ~For
  example }{C:/eclipse}
\end{itemize}

\hypertarget{h.3dy6vkm}{\section{\texorpdfstring{{GNU ARM Eclipse
Plug-ins}}{GNU ARM Eclipse Plug-ins}}\label{h.3dy6vkm}}

{To install these plug-ins use the Eclipse standard install/update
mechanism}

\begin{itemize}
\tightlist
\item
  {run the Eclipse IDE by selecting the }{eclipse.exe}{~from the
  installation directory. Ex: ~}{C:/eclipse/eclipse.exe}
\item
  {select the suggested default workspace (just click in the
  }{OK}{~button). We will select another workspace to compile the demo
  application later.}
\item
  {in the Eclipse menu select }{Help--\textgreater{}Install New
  Software\ldots{}}
\item
  {in the }{Install}{~window, click the }{Add\ldots{}}{~button}
\item
  {in the }{Add Repository}{~dialog box fill in the fields:}
\end{itemize}

\begin{itemize}
\tightlist
\item
  {Name: }{GNU ARM Eclipse Plug-ins}
\item
  {Location: }{http://gnuarmeclipse.sourceforge.net/updates}
\end{itemize}

{\includegraphics{images/image6.png}}

\begin{itemize}
\tightlist
\item
  {click the OK button}
\item
  {the main window should list a group named CDT GNU Cross Development
  Tools; expand it}
\item
  {select the ``GNU ARM C/C++ Cross Compile'' plug-in}
\end{itemize}

{\includegraphics{images/image9.png}}

\begin{itemize}
\tightlist
\item
  {click the Next button and follow the usual installation procedure}
\end{itemize}

\hypertarget{h.1t3h5sf}{\section{\texorpdfstring{{C/C++ GDB Hardware
Debugging
Plug-in}}{C/C++ GDB Hardware Debugging Plug-in}}\label{h.1t3h5sf}}

\begin{itemize}
\tightlist
\item
  {In the Eclipse menu select Help --\textgreater{}Install New
  Software\ldots{}}
\item
  {in the Work with drop down select Mars -
  http://download.eclipse.org/releases/mars (if your Eclipse version is
  not the Mars, you should find a similar entry with the corresponding
  version).}
\item
  {the main window should list a group named Mobile and Device
  Development; expand it}
\item
  {select the C/C++ GDB Hardware Debugging plug-in}
\end{itemize}

{\includegraphics{images/image8.png}}

\begin{itemize}
\tightlist
\item
  {click the Next button and follow the usual installation procedure}
\end{itemize}

\hypertarget{h.4d34og8}{\section{\texorpdfstring{{STM32 ST-LINK
Utility}}{STM32 ST-LINK Utility}}\label{h.4d34og8}}

{The STM32 ST-LINK Utility software facilitates fast in-system
programming of the STM32 microcontroller families in development
environments via the tools, ST-LINK and ST-LINK/V2. It will be used to
load the }{ThinkOS Bootloader}{~on the target board.}

\begin{itemize}
\tightlist
\item
  {Download }{STSW-LINK004}{~(STM32 ST-LINK utility), the downloaded
  file is called }{stsw-link004.zip}{. The link:
  }{\href{https://www.google.com/url?q=http://www.st.com/web/catalog/tools/FM147/SC1887/PF258168\&sa=D\&ust=1511275046378000\&usg=AFQjCNG92HuOtBhSNJxbtqHYuxedmOi_CA}{http://www.st.com/web/catalog/tools/FM147/SC1887/PF258168}}{~was
  pointing to the file at the time of this writing. Alternatively you
  can search for: }{STM32 ST-LINK utility}{~in the STMicroelectronics
  website:
  }{\href{https://www.google.com/url?q=http://www.st.com\&sa=D\&ust=1511275046378000\&usg=AFQjCNFcwsIyrxx1y2iQSZFPc5tU-A4JIQ}{http://www.st.com}}
\item
  {Decompress the }{stsw-link004.zip}{~file in a temporary folder and
  execute the installer: }{STM32 ST-LINK Utility\_v3.7.0.exe}
\end{itemize}

\hypertarget{h.2s8eyo1}{\section{\texorpdfstring{{ThinkOS Source
Code}}{ThinkOS Source Code}}\label{h.2s8eyo1}}

\begin{itemize}
\tightlist
\item
  {Get the }{ThinkOS}{~source code from GitHub }{YARD-ICE}{~project:
  }{\href{https://www.google.com/url?q=https://github.com/bobmittmann/yard-ice/archive/0.24.zip\&sa=D\&ust=1511275046379000\&usg=AFQjCNH4WK30jeFdsIOws19RGCXF5Ekb5Q}{https://github.com/bobmittmann/yard-ice/archive/0.24.zip}}
\item
  {Extract the source code into a local directory, ex.:
  }{C:\textbackslash{}devel}
\end{itemize}

\subsection{\texorpdfstring{{Cloning the Git
repository}}{Cloning the Git repository}}\label{cloning-the-git-repository}

{Alternatively you can clone the Git repository from:
}{\href{https://www.google.com/url?q=https://github.com/bobmittmann/yard-ice\&sa=D\&ust=1511275046380000\&usg=AFQjCNFusJgGgAORHVsXadKiiwWkzF5HBQ}{https://github.com/bobmittmann/yard-ice}}

\hypertarget{h.17dp8vu}{\section{\texorpdfstring{{Importing Eclipse
Projects}}{Importing Eclipse Projects}}\label{h.17dp8vu}}

{To compile the }{ThinkOS BootLoader}{~and the demo application you must
import the Eclipse projects existing in the ThinkOS source code tree
into your Eclipse workspace.}

\begin{itemize}
\tightlist
\item
  {Select }{board/stm32f3discovery }{as your Eclipse workspace:
  `}{C:\textbackslash{}devel\textbackslash{}yard-ice\textbackslash{}src\textbackslash{}board\textbackslash{}stm32f3discovery}{'}
\item
  {In the Eclipse menu select }{File--\textgreater{}Import\ldots{}}
\item
  {in the }{Import}{~window, expand the }{General}{~group}
\item
  {Select }{Existing Projects into Workspace}{~and click }{Next
  \textgreater{}}
\end{itemize}

{\includegraphics{images/image11.png}}

\begin{itemize}
\tightlist
\item
  {check `}{Select root directory}{'}
\item
  {select the current workspace directory: Ex.:
  `}{C:\textbackslash{}devel\textbackslash{}yard-ice\textbackslash{}src\textbackslash{}board\textbackslash{}stm32f3discovery}{'}
\end{itemize}

{\includegraphics{images/image10.png}}

\begin{itemize}
\tightlist
\item
  {click the ``Finish'' button }
\end{itemize}

\hypertarget{h.3rdcrjn}{\section{\texorpdfstring{{Compiling the ThinkOS
Bootloader}}{Compiling the ThinkOS Bootloader}}\label{h.3rdcrjn}}

\begin{itemize}
\tightlist
\item
  {In the }{Project Explorer}{~tab select the }{boot}{~project. If the
  Project Explorer tab/window is not visible select }{Window}{→}{Show
  View}{→}{Project Explorer}{~from the menu.}
\item
  {Press the build button (Hammer) or select
  }{Project--\textgreater{}Build Project}
\end{itemize}

{\includegraphics{images/image14.png}}

\subsection{\texorpdfstring{{Using the Command
Shell}}{Using the Command Shell}}\label{using-the-command-shell}

{It is not necessary to have the Eclipse environment to compile the
projects in the source tree, you could use the windows Command Prompt or
a MinGW/MSYS terminal directly.}

\begin{itemize}
\tightlist
\item
  {Open a windows shell: Start}{→}{All
  Programs}{→}{Accessories}{→}{Command Prompt}
\item
  {Change to the directory
  src\textbackslash{}board\textbackslash{}stm32f3discovery\textbackslash{}boot
  in the yard-ice source tree.}
\item
  {Type: }{make}
\end{itemize}

{\includegraphics{images/image12.png}}

{}

\hypertarget{h.26in1rg}{\section{\texorpdfstring{{Compiling the board
SDK libraries}}{Compiling the board SDK libraries}}\label{h.26in1rg}}

\begin{itemize}
\tightlist
\item
  {In the }{Project Explorer}{~tab select the }{sdklibs}{~project.}
\item
  {Press the build button (Hammer) or select
  }{Project--\textgreater{}Build Project}
\end{itemize}

\hypertarget{h.lnxbz9}{\section{\texorpdfstring{{Compiling the demo
application}}{Compiling the demo application}}\label{h.lnxbz9}}

\begin{itemize}
\tightlist
\item
  {In the }{Project Explorer}{~tab select the }{led\_test}{~project}
\item
  {Press the build button (Hammer) or select
  }{Project--\textgreater{}Build Project}
\end{itemize}

\hypertarget{h.35nkun2}{\section{\texorpdfstring{{STM32F3Discovery
board}}{STM32F3Discovery board}}\label{h.35nkun2}}

{The }{STM32F3Discovery}{~is a low cost development board for the STM32
F3 series Cortex-M4 mixed-signals. It includes everything required for
beginners and experienced users to get started quickly. Based on the
}{STM32F303VCT6}{, it includes an ST-LINK/V2 embedded debug tool,
accelerometer, gyroscope and e-compass ST MEMS, USB connection, LEDs and
pushbuttons.}

{\includegraphics{images/image13.jpg}}

{The Stm32f3Discovery board has 2 USB connectors:}

\begin{itemize}
\tightlist
\item
  {One at the center of the board marked as }{USB ST-LINK}{~which is
  connected to an auxiliary chip who in turn connects to the JTAG/SWD
  interface of STM303 MPU. We will use this interface to program the
  Boot-Loader once.}
\item
  {The second one, located at the side of the board, marked as }{USB
  USER}{, is attached directly to the STM32F303 MPU. The ThinkOS
  Debug-Monitor will make use of this port as a console for user
  interface and for remote debugging through GDB.}
\end{itemize}

\hypertarget{h.1ksv4uv}{\section{\texorpdfstring{{ThinkOS on
STM32F3Discovery}}{ThinkOS on STM32F3Discovery}}\label{h.1ksv4uv}}

{The ThinkOS can be configured to run as a Boot-Loader and
Debug-Monitor, this way the OS will be resident in the board adding the
support for Application (Firmware) updates as well as debug support.}

{To install or upgrade the ThinkOS in your Stm32f3Discovery board follow
these instructions:}

\begin{itemize}
\tightlist
\item
  {Compile the }{boot}{~project }
\item
  {Install the STM32 ST-LINK Utility}
\item
  {Connect a USB cable to the USB ST-LINK connector in the board and
  into your PC, this will power the board.}
\item
  {Run the STM32 ST-LINK Utility}
\item
  {Connect the program to the board by clicking the connect button or by
  selecting }{Target}{→}{Connect}{~from the menu.}
\item
  {Open the }{thinkos.bin}{~file compiled at step 1, located at
  src\textbackslash{}board\textbackslash{}stm32f3discovery\textbackslash{}boot\textbackslash{}release.
  Use the }{File}{→}{Open file\ldots{}}{~menu.}
\item
  {Program the board by selecting }{Target}{→}{Program\ldots{}}{~and
  pressing the }{Start}{~button on the }{Download {[} thinkos.bin
  {]}}{~dialog box.}
\end{itemize}

{\includegraphics{images/image15.png}}

\begin{itemize}
\tightlist
\item
  {After the programming completed the LEDs on the circular pattern in
  the board will light up in a spinning pattern.}
\end{itemize}

\hypertarget{h.44sinio}{\subsubsection{\texorpdfstring{{USB Virtual COM
Port}}{USB Virtual COM Port}}\label{h.44sinio}}

{You can now remove the mini USB from the central connector (}{USB
ST-LINK}{) and insert it into the adjacent connector at the side of the
board (}{USB USER}{). At this point a virtual serial port driver will be
installed in your system. If the driver fails to install automatically
you need to install it manually. }

\begin{itemize}
\tightlist
\item
  {Download the STSW-STM32102STM32 Virtual COM Port Driver from the
  STMicroelectonics website:
  }{\href{https://www.google.com/url?q=http://www.st.com/web/en/catalog/tools/PF257938\&sa=D\&ust=1511275046387000\&usg=AFQjCNHybx7g4II_reWXNPTvvvAeYlCa4Q}{http://www.st.com/web/en/catalog/tools/PF257938}}{.
  The file should be `stsw-stm32102.zip'}
\item
  {Decompress the file into a temporary folder and run the installation
  program: }{VCP\_V1.4.0\_Setup.exe}{.}
\end{itemize}

{If the driver is installed the Devices and Printers list will show it:
Start}{→Devices and Printers. The device will be displayed in the
}{Unspecified}{~section and is called `}{STMicroelectronics Virtual COM
Port}{'. If you select the device in the list, more details will be
displayed at the bottom. The Model should read: `}{ThinkOS Debug
Monitor}{'.}

{\includegraphics{images/image16.png}}

\hypertarget{h.2jxsxqh}{\section{\texorpdfstring{{ThinkOS
Eclipse/GDB}}{ThinkOS Eclipse/GDB}}\label{h.2jxsxqh}}

{The }{ThinkOS Debug-Monitor}{~is very high priority application
embedded in the }{ThinkOS BootLoader }{which provides a link between the
host GDB debugger and the ThinkOS Kernel, allowing any front end capable
of using GDB for debugging to load and debug Application (Firmware) on
the target board.}

{The }{ThinkOS Debug-Monitor}{~emulates a serial port over the USB to
communicate with the GDB (debugger) running in the PC, you need to make
sure }{ST USB VCOM Port driver}{~is correctly installed in your system
in order to debug }{ThinkOS}{~programs on the Stm32f3Discovery board.}

{This section describes how to setup a remote GDB session using Eclipse
CDT and a }{Stm32f3Discovery}{~board running }{ThinkOS}{.}

\begin{itemize}
\tightlist
\item
  {Make sure the }{Stm32f3Discovery}{~board is attached to the }{USB
  USER}{~interface.}
\item
  {Check the }{Virtual COM Port}{~number. Go to }{Start}{→}{Devices and
  Printers}{~and locate the `}{STMicroelectronics Virtual COM Port}{'.
  The COM port number will be displayed just under the device.}
\item
  {Open the Eclipse project named }{led\_test}{~in the stm32f3discovery
  workspace.}
\item
  {Build the project.}
\item
  {Open the Debug Configurations window selecting }{Run}{→}{Debug
  Configurations\ldots{}}{~from the menu.}
\item
  {Double click the item }{GDB Hardware Debugging}{~or press the `New'
  button.}
\item
  {Select the Debugger tab and fill in the information accordingly:}
\end{itemize}

\begin{itemize}
\tightlist
\item
  {GDB Command: `arm-none-eabi-gdb -b 38400' (please notice the space
  between the --b and the 38400)}
\item
  {JTAG Device: select Generic Serial}
\item
  {GDB Connection String:
  `\textbackslash{}\textbackslash{}.\textbackslash{}COM14'. ~Use here
  the COM port reported in the }{Devices and Printers}{~list.}
\end{itemize}

{\includegraphics{images/image17.png}}

\begin{itemize}
\tightlist
\item
  {Click the `Apply' button}
\item
  {Click the `Debug' button}
\item
  {The Eclipse will change the perspective to Debug and the Debug
  toolbar should appear.}
\item
  {Press the Resume (F8) button (yellow green arrow) to run the
  program.}
\item
  {The LEDs should flash in sequence.}
\item
  {Stop the program by clicking in the pause button. This will list the
  threads of the system and the source code window will indicate the
  current execution line.}
\item
  {At this point you can perform step-by-step execution, insert
  breakpoints, inspect variables, expression registers etc.}
\item
  {When you finish debugging you can close the debug session by pressing
  the Disconnect button (red zigzag). Before starting a new section
  always disconnect the previous one or the GDB will complain it can't
  open the serial/USB port.}
\end{itemize}

{\includegraphics{images/image18.png}}

\section{\texorpdfstring{{Summary of Packages and
Files}}{Summary of Packages and Files}}\label{summary-of-packages-and-files}

\protect\hypertarget{t.36dcbdc507fe00169c271aaebe6543bb601998bb}{}{}\protect\hypertarget{t.0}{}{}

\begin{longtable}[]{@{}ll@{}}
\toprule
\begin{minipage}[t]{0.47\columnwidth}\raggedright\strut
{File}\strut
\end{minipage} & \begin{minipage}[t]{0.47\columnwidth}\raggedright\strut
{Content}\strut
\end{minipage}\tabularnewline
\begin{minipage}[t]{0.47\columnwidth}\raggedright\strut
{stsw-link009.zip }\strut
\end{minipage} & \begin{minipage}[t]{0.47\columnwidth}\raggedright\strut
{ST-LINK USB driver ~ ~ ~ ~ ~ ~ ~ }\strut
\end{minipage}\tabularnewline
\begin{minipage}[t]{0.47\columnwidth}\raggedright\strut
{stsw-link004.zip}\strut
\end{minipage} & \begin{minipage}[t]{0.47\columnwidth}\raggedright\strut
{STM32 ST-LINK utility}\strut
\end{minipage}\tabularnewline
\begin{minipage}[t]{0.47\columnwidth}\raggedright\strut
{stlinkupgrade.zip }\strut
\end{minipage} & \begin{minipage}[t]{0.47\columnwidth}\raggedright\strut
{ST-LINK firmware upgrade}\strut
\end{minipage}\tabularnewline
\begin{minipage}[t]{0.47\columnwidth}\raggedright\strut
{eclipse-cpp-mars-R-win32-x86\_64.zip}\strut
\end{minipage} & \begin{minipage}[t]{0.47\columnwidth}\raggedright\strut
{Eclipse IDE for C/C++ Developers}\strut
\end{minipage}\tabularnewline
\begin{minipage}[t]{0.47\columnwidth}\raggedright\strut
{jre-8u60-windows-x64.exe}\strut
\end{minipage} & \begin{minipage}[t]{0.47\columnwidth}\raggedright\strut
{Java Runtime Environement (JRE)}\strut
\end{minipage}\tabularnewline
\begin{minipage}[t]{0.47\columnwidth}\raggedright\strut
{gcc-arm-none-eabi-4\_9-2015q2-20150609-win32.exe}\strut
\end{minipage} & \begin{minipage}[t]{0.47\columnwidth}\raggedright\strut
{GNU Toolchain for ARM Processros}\strut
\end{minipage}\tabularnewline
\begin{minipage}[t]{0.47\columnwidth}\raggedright\strut
{make-3.81.exe}\strut
\end{minipage} & \begin{minipage}[t]{0.47\columnwidth}\raggedright\strut
{GNU Make for Windows}\strut
\end{minipage}\tabularnewline
\begin{minipage}[t]{0.47\columnwidth}\raggedright\strut
{coreutils-5.3.0.exe}\strut
\end{minipage} & \begin{minipage}[t]{0.47\columnwidth}\raggedright\strut
{GNU CoreUtils for Windows}\strut
\end{minipage}\tabularnewline
\begin{minipage}[t]{0.47\columnwidth}\raggedright\strut
{yard-ice-0.23.zip}\strut
\end{minipage} & \begin{minipage}[t]{0.47\columnwidth}\raggedright\strut
{ThinkOS Source Code}\strut
\end{minipage}\tabularnewline
\begin{minipage}[t]{0.47\columnwidth}\raggedright\strut
{stsw-stm32102.zip}\strut
\end{minipage} & \begin{minipage}[t]{0.47\columnwidth}\raggedright\strut
{STSW-STM32102STM32 Virtual COM Port Driver}\strut
\end{minipage}\tabularnewline
\bottomrule
\end{longtable}

\section{\texorpdfstring{{References}}{References}}\label{references}

\subsubsection{\texorpdfstring{{Gnu Tools for
Window}}{Gnu Tools for Window}}\label{gnu-tools-for-window}

{More information on the GNU Make and CoreUtils package can be found at:
}

\begin{itemize}
\tightlist
\item
  {\href{https://www.google.com/url?q=http://gnuwin32.sourceforge.net/packages/make.htm\&sa=D\&ust=1511275046397000\&usg=AFQjCNHQhDY13EJKv3bQUa54ST0nRARnLg}{http://gnuwin32.sourceforge.net/packages/make.htm}}
\item
  {\href{https://www.google.com/url?q=http://gnuwin32.sourceforge.net/packages/coreutils.htm\&sa=D\&ust=1511275046397000\&usg=AFQjCNHzCBMuNsNxheAyqUiqalsOFlTK9g}{http://gnuwin32.sourceforge.net/packages/coreutils.htm}}
\end{itemize}

\subsubsection{\texorpdfstring{{GNU ARM Eclipse
Plug-Ins}}{GNU ARM Eclipse Plug-Ins}}\label{gnu-arm-eclipse-plug-ins}

{Quick info on GNU ARM Eclipse Plug-Ins:}

\begin{itemize}
\tightlist
\item
  {Name: `}{GNU ARM Eclipse Plug-ins}{'}
\item
  {Location:
  }{\href{https://www.google.com/url?q=http://gnuarmeclipse.sourceforge.net/updates\&sa=D\&ust=1511275046399000\&usg=AFQjCNEdmX4d-t5pswZkr32N4e9t7J324Q}{http://gnuarmeclipse.sourceforge.net/updates}}
\item
  {Website:
  }{\href{https://www.google.com/url?q=http://gnuarmeclipse.livius.net/blog\&sa=D\&ust=1511275046399000\&usg=AFQjCNHwOOZ7pOFg21RgjhMN2SMiP27Adw}{http://gnuarmeclipse.livius.net/blog}}
\end{itemize}

\subsubsection{\texorpdfstring{{ThinkOS}}{ThinkOS}}\label{h.ajl6qmj6j511}

{The ThinkOS source is part of the YARD-ICE project.}

{\includegraphics{images/image2.png}}

\begin{itemize}
\tightlist
\item
  {YARD-ICE Project Location:
  }{\href{https://www.google.com/url?q=https://github.com/bobmittmann/yard-ice\&sa=D\&ust=1511275046400000\&usg=AFQjCNGufoXP2wfG8botuTmr2t7ckLYnvQ}{https://github.com/bobmittmann/yard-ice}}
\item
  {Latest Release:
  }{\href{https://www.google.com/url?q=https://github.com/bobmittmann/yard-ice/archive/0.24.zip\&sa=D\&ust=1511275046401000\&usg=AFQjCNF-nzx-oGprzrbxGIY-uiZXVNkATg}{https://github.com/bobmittmann/yard-ice/archive/0.24.zip}}
\item
  {Git Repository (HTTPS)}{:
  }{https://github.com/bobmittmann/yard-ice.git}
\item
  {Git Repository (SSH): }{git@github.com:bobmittmann/yard-ice.git}
\end{itemize}

{}
